\documentclass[10pt]{article}

% Lines beginning with the percent sign are comments
% This file has been commented to help you understand more about LaTeX

% DO NOT EDIT THE LINES BETWEEN THE TWO LONG HORIZONTAL LINES

%---------------------------------------------------------------------------------------------------------

% Packages add extra functionality.
\usepackage{
	times,
	graphicx,
	epstopdf,
	fancyhdr,
	amsfonts,
	amsthm,
	amsmath,
	algorithm,
	algorithmic,
	xspace,
	hyperref}
\usepackage[left=1in,top=1in,right=1in,bottom=1in]{geometry}
\usepackage{sect sty}	%For centering section headings
\usepackage{enumerate}	%Allows more labeling options for enumerate environments 
\usepackage{epsfig}
\usepackage[space]{grffile}
\usepackage{booktabs}
\usepackage{amsmath}
\usepackage[super]{nth}

% This will set LaTeX to look for figures in the same directory as the .tex file
\graphicspath{.} % The dot means current directory.

\pagestyle{fancy}

\lhead{\YOURID}
\chead{\MyLang: Language Specification}
\rhead{\today}
\lfoot{CSCI 334: Principles of Programming Languages}
\cfoot{\thepage}
\rfoot{Spring 2020}

% Some commands for changing header and footer format
\renewcommand{\headrulewidth}{0.4pt}
\renewcommand{\headwidth}{\textwidth}
\renewcommand{\footrulewidth}{0.4pt}

% These let you use common environments
\newtheorem{claim}{Claim}
\newtheorem{definition}{Definition}
\newtheorem{theorem}{Theorem}
\newtheorem{lemma}{Lemma}
\newtheorem{observation}{Observation}
\newtheorem{question}{Question}

\setlength{\parindent}{0cm}
%---------------------------------------------------------------------------------------------------------

% DON'T CHANGE ANYTHING ABOVE HERE
\begin{document}

% Edit below as instructed
\newcommand{\YOURID}{Josiel Aponte}	% Replace "Your Name Here" with your name
\newcommand{\MyLang}{DTM}	% Replace MyLang with your language name #
\newcommand{\ProblemHeader}	% Don't change this!

\vspace{\baselineskip}	% Add some vertical space

% Refer to the lab handouts to determine what should go in each of these sections.  Each lab is additive. So lab 8 should include everything you wrote in lab 7.  Lab 9 should include everything you wrote in lab 8, etc.

\begin{center}
    \Large\textbf{A Look into DTM (Do the Math)}\\
\end{center}

\section{\textit{Introduction}}

% What problem does your language solve? Why does this problem need its own programming
% language?

The \textit{Do the Math} (DTM) programming language is a language that can be used to 
solve math problems. Solvable problems are limited to those including the following 
operations: addition, subtraction, multiplication, division, or exponentiation. While this
may not necessarily need its own programming language, DTM serves as a way to reinforce 
the order of operation rules and avoid problems that are open to interpretation in how 
they should be solved. An example of this is a popular math problem that makes its way 
around social media occasionally, causing people to second guess the way they learned how 
to do math: $6\div2(1+2)$. Possible solutions to this problem include 1 or 9, depending on
which operations are done first. However, with DTM, there would be no room left for 
interpretation because the language requires specific grouping of numbers in order to 
solve a problem. Thus, DTM clears any confusion there may be when solving a math problem 
and serves as a way to help people learn math.

\section{\textit{Design Principles}}

% Languages can solve problems in many ways. What are the guiding aesthetic or technical 
% principles that underpin its design?

As mentioned, a key part of DTM is leaving no room left for interpretation in what order a
math problem should be solved in. This is done by having all possible operations between 
two numbers be enclosed in parentheses. An example for this is as follows: suppose you 
would like to divide some $i$, $j$, and $k$, which in its simplest form would be $i \div j
\div k$. Following the order of operations, in which division is done from left to right, 
the problem expressed in terms of operations between two numbers would be $((i \div j) 
\div k)$, which makes it clear that $i$ is divided by $j$, and their result is divided by 
$k$. Where the rules get confusing is in the involvement of parentheses taking the place 
for multiplication, where $i(j) = i \cdot j$. As can be seen in the example problem of the
introduction, it causes issues because parentheses are supposed to take precedence of all 
other operations. However, since it can sometimes imply multiplication when in the form 
$i(j)$, in an example such as $6\div2(1+2)$, after evaluating $(1+2)$ as $3$, it isn't 
clear if $6\div2$ or if $2(3)$ should be done first. \\

Another way to combat the confusion of order of operations is with the inclusion of prefix
notation in DTM. The examples shown previously were all in infix notation, which takes the
form of <number> <operation> <number>. However, as the name suggests, prefix notation 
evaluates a math problem with the operation being done preceding the values, following the
form <operation> <number> <number>. This helps prevent confusion as there is no need to 
know the order of operation rules and how they coexist with another. The operation is just
completed on all of the numbers that follow it and in the case of multiple operations, 
they are done from right to left, as the left-most operation relies on the values to the 
right of it to be done. Prefix notation also includes the advantage of being able to 
provide a list of numbers with a length greater than 2 after the operation, and have it 
all be evaluated in the same way without repetition of the operation. In other words, the 
infix expression $((((((a + b) + c) + d) + e) + f) + g)$ would just be $(+$ $a$ $b$ $c$ 
$d$ $e$ $f$ $g)$ in prefix.

\section{\textit{Example Programs}}

% Provide three working example programs in your language. If any of the examples from
% your proposal do not yet work, please either extend the language to support them or 
% replace them with working examples. Explain exactly how to run each example (e.g., 
% dotnet run example-1.lang) and what the expected output should be (e.g., 2).

Here is a list of example problems with their expected output. Each of the programs are included in a directory titled, "examples" in the source code and can be tested using the command $\texttt{dotnet run "../examples/example-<number>.dtm"}$, when in the $\tt{lang}$ directory of the source code. 

\begin{enumerate}
    \item $\texttt{(/ (* (+ 1 2) 3) 4)} = 2.25$
    \item $\texttt{(((1 + 2) * 3) / 4)}= 2.25$
    \item $\texttt{(- 65 (* 40 (\^{} 3 2)))} = -295.0$
    \item $\texttt{(((5 * (76 \^{} 2)) - 40) + 9)} = 28849.0$
    \item $\texttt{(1004 - ((29 * 85) / 3))} = 182.\overline{3}$
    \item $\texttt{(+ (\^{} 6353 -3) (* (/ 73637 7373) 2))} = 19.97477282$
\end{enumerate}

In addition to the examples provided in the folder, the DTM language can be tested with unique input, as long as it fits into the form of infix or postfix notation. It can be tested when in the $\tt{lang}$ directory, by typing the following into the command line: $\texttt{dotnet run "<valid-expression>"}$. 

\section{\textit{Language Concepts}}

% What are the core concepts a user needs to understand in order to write programs? Think 
% in terms of both “primitives” and “combining forms.” What are the key ideas and how are 
% they combined?

In terms of primitives, there are only 2 to really be aware of, a number (<num>) and a decimal (<deci>). Both are represented as floats, which allows outcomes that may result in a float to be returned. With these two primitives, the rest of the available expressions can be formed. There are two type of expressions to be aware of, infix expressions and prefix expressions. An infix expression is in the form of numbers (or decimals) that focuses on evaluating a number and an expression together (refer to syntax for clarification). A prefix expression is in the form of numbers (or decimals) that focuses on evaluating a list of numbers together (refer to syntax for clarification). Both infix and prefix expressions rely on 5 operators, +, -, *, /, and $\texttt{\^{}}$. Another key part of the language is that each possible expression is separated by a white-space, which in most cases is a space. Lastly, all possible expressions being joined by operators must be in a set of parentheses, as this prevents any sort of confusion that may occur from order of operation rules.

\section{\textit{Syntax}}

% Provide a formal syntax for all supported operations, written in Backus-Naur form. This
% documentation should provide all of the rules necessary for a user to generate a valid 
% program.

\begin{verbatim}
    <expr>      ::= (<num> <ws> <op> <ws> <expr>)
                | (<op> <ws> <expr> <ws> <expr>+)
                | <num>              
    <num>       ::= <d><num>
                | <d>
    <deci>      ::= <num><dot><num>
    <d>         ::= 0 | 1 | 2 | 3 | 4 | 5 | 6 | 7 | 8 | 9
    <op>        ::= + | - | * | / | ^ 
    <dot>       ::= .
    <ws>        ::= ␣ 
\end{verbatim}

\newpage

\section{\textit{Semantics}}

% Update the semantics section from the previous assignment to explain all of your 
% currently supported data types and operations. This section should explain how a user 
% understands the effect of a syntactic construct given in the formal syntax section. This % need not be so detailed that it explains what the code does; instead it should explain 
% what the syntax means. In other words, focus on what each language element achieves 
% instead of explaining how it does it. Please refer to the example shown in the previous 
% assignment for guidance. Your semantics section need not be in the tabular form shown if % a table is inconvenient.

\begin{table}[h!]
  \begin{center}
    \label{tab:table1}
    \begin{tabular}{p{.13\textwidth}|p{.2\textwidth}|p{.23\textwidth}|p{.12\textwidth}|p{.26\textwidth}}
      \textbf{Syntax} & \textbf{Abstract Syntax} & \textbf{Type} & \textbf{Prec./Assoc.} & \textbf{Meaning}\\
      \hline
      $n$ & Num of float & $float$ & n/a & n is a primitive. We represent integers using the F\# float data type.\\
      \hline
      $d$ & Deci of float & $float$ & n/a & d is a primitive. We represent decimals using the F\# float data type.\\
      \hline
      $(p_1 + p_2)$ & In\_Plus of Expr * Expr & $float \rightarrow float \rightarrow float$ & 3/left & In\_Plus evaluates in infix form $p_1$ and $p_2$, adding their results, yielding a float. \\
      \hline
      $(s_1 - s_2)$ & In\_Sub of Expr * Expr & $float \rightarrow float \rightarrow float$ & 3/left & In\_Sub evaluates in infix form $s_1$ and $s_2$, subtracting their results, yielding a float. \\
      \hline
      $(m_1 * m_2)$ & In\_Mult of Expr * Expr & $float \rightarrow float \rightarrow float$ & 2/left & In\_Mult evaluates in infix form $m_1$ and $m_2$, multiplying their results, yielding a float.\\
      \hline
      $(d_1 / d_2)$ & In\_Div of Expr * Expr & $float \rightarrow float \rightarrow float$ & 2/left & In\_Div evaluates in infix form $d_1$ and $d_2$, dividing their results, yielding a float.\\
      \hline
      $(e_1$ $\hat{\mkern6mu}$ $e_2)$ & In\_Expo of Expr * Expr & $float \rightarrow float \rightarrow float$ & 1/left & In\_Expo evaluates in infix form $e_1$ and $e_2$, raising $e_1$ to the power of $e_2$, yielding a float.\\
      \hline
      $(+$ $p_1$ $p_2\dots)$ & Pre\_Plus of Expr list & $float \rightarrow float \rightarrow float$ & 3/left & In\_Plus evaluates in prefix form a list of $p$'s, adding their results, yielding a float. \\
      \hline
      $(-$ $s_1$ $s_2\dots)$ & Pre\_Sub of Expr list & $float \rightarrow float \rightarrow float$ & 3/left & In\_Sub evaluates in prefix form a list of $s$'s, subtracting their results, yielding a float. \\
      \hline
      $(*$ $m_1$ $m_2\dots)$ & Pre\_Mult of Expr list & $float \rightarrow float \rightarrow float$ & 2/left & In\_Mult evaluates in prefix form a list of $m$'s, multiplying their results, yielding a float.\\
      \hline
      $(/$ $d_1$ $d_2\dots)$ & Pre\_Div of Expr list & $float \rightarrow float \rightarrow float$ & 2/left & In\_Div evaluates in prefix form a list of $d$'s, dividing their results, yielding a float.\\
      \hline
      $(\hat{\mkern6mu}$ $e_1$ $e_2\dots)$ & Pre\_Expo of Expr list & $float \rightarrow float \rightarrow float$ & 1/left & In\_Expo evaluates in prefix form a list of $e$'s, raising $e_1$ to the power of $e_2$ and so on, yielding a float.\\
    \end{tabular}
  \end{center}
\end{table}

\newpage

\section{\textit{Remaining Work}}

% Add a section at the end of your specification that explains which features are not yet
% implemented but which you plan to implement by the final project deadline. This should 
% include any essential remaining data types and operations described in your proposal 
% that you have not yet implemented.

% DO NOT DELETE ANYTHING BELOW THIS LINE
\end{document}